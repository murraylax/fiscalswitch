\documentclass{beamer}
\usepackage{beamerthemeshadow}
\usepackage{verbatim}

\usepackage{lastpage}
\usepackage{xcolor}
\usepackage{pgf}
\usepackage{colortbl}
\usepackage{hyperref}
\usepackage{multirow}

\usepackage{siunitx}
\sisetup{input-symbols=(), group-digits  = false} 

\newcommand{\bi}{\begin{itemize}}
\newcommand{\ei}{\end{itemize}}
\newcommand{\be}{\begin{enumerate}}
\newcommand{\ee}{\end{enumerate}}
\newcommand{\bd}{\begin{description}}
\newcommand{\ed}{\end{description}}
\newcommand{\prbf}[1]{\textbf{#1}}
\newcommand{\prit}[1]{\textit{#1}}
\newcommand{\beq}{\begin{equation}}
\newcommand{\eeq}{\end{equation}}
\newcommand{\bdm}{\begin{displaymath}}
\newcommand{\edm}{\end{displaymath}}

\newcommand{\ft}[1]{
  \frametitle{\begin{tabular}{p{4.2in}r} \textcolor{white}{#1} & \small{\insertframenumber / \inserttotalframenumber} \end{tabular}}
  \setbeamercovered{transparent=18}
}

\newcommand{\eft}[1]{
  \frametitle{\begin{tabular}{p{4in}r} \textcolor{white}{#1} & \small{\hyperlink{f:questions}{\beamergotobutton{GO BACK}}} \end{tabular}}
  \setbeamercovered{transparent=18}
}

\newcommand{\stepinv}{\setbeamercovered{invisible}}
\newcommand{\stopinv}{\setbeamercovered{transparent=18}}
\newcommand{\uncoverinv}[1]
{
  \setbeamercovered{invisible}
  \uncover<+->{#1}
  \setbeamercovered{transparent=18}
}
\newcommand{\ans}[1]{\textcolor{blue}{#1}}
\newcommand{\ansinv}[1]
{
  \setbeamercovered{invisible}
  \uncover<+->{\textcolor{blue}{#1}}
  \setbeamercovered{transparent=18}
}
\newcommand{\setinv}{\setbeamercovered{invisible}}
\newcommand{\setvis}{\setbeamercovered{transparent=18}}
\newcommand{\centerpic}[2]
{
  \begin{center}
  \includegraphics[#1]{#2}
  \end{center}
}
\newcommand{\h}[1]{\hat{#1}}
\newcommand{\ds}{\displaystyle}

\definecolor{light}{rgb}{1.0,0.7,0.7}
\definecolor{BrickRed}{rgb}{0.8,0.1,0.1}
%\definecolor{light}{rgb}{1.0,0.5,0.5}
\newcommand{\hl}[1]{\only<#1>{\cellcolor{light}}}

\definecolor{mycolor}{rgb}{0.6,0.0,0.0}
\usecolortheme[named=mycolor]{structure}

\title{Regime Switching in Fiscal Policy Composition}
\author[James Murray, University of Wisconsin - La Crosse]
{
James Murray\\
Department of Economics\\
University of Wisconsin - La Crosse
}
\date{August 6, 2016}

\begin{document}

\frame{\titlepage \setcounter{framenumber}{0}}

\begin{frame}
  \ft{Purpose}
  \uncover<+-> {
  \begin{block}{Describe fiscal policy dynamics}
    \begin{itemize}
    \item Income tax rate
    \item Net transfer payments
    \item Government expenditures
    \item Deficits
    \end{itemize}
  \end{block}
  } % end uncover

  \uncover<+->{
  \begin{block}{Describe debt service}
    \begin{enumerate}
    \item How do these fiscal policy variables respond to \textit{debt / GDP}?
    \item What is the implied target for \textit{debt / GDP}?
    \item Is there switching in his behavior?
    \end{enumerate}
  \end{block}
  } % end uncover

  \uncover<+->{
  \begin{block}{Describe stabilizing behavior}
    \begin{enumerate}
    \item How do fiscal policy variables respond to \textit{output gap}?
    \item Is there switching in this behavior?
    \end{enumerate}
  \end{block}
  } % end uncover
\end{frame}

\begin{frame}
  \ft{Fiscal dynamics matter}
  \uncover<+-> {
  \begin{block}{Debt target and tax response matter}
    Given \textit{smaller debt/GDP target} and/or \textit{larger response of tax rate},
    \begin{itemize}
    \item People expect higher income taxes $\rightarrow$ decreases consumption, investment, real GDP.
    \item Similar to Richter and Throckmorton (EER, 2015)
    \end{itemize}
  \end{block}
  } % end uncover

  \uncover<+-> {
  \begin{block}{Fiscal composition matters}
    Leeper, Plante, and Traum (JoE, 2010)
    \begin{itemize}
    \item Rich set of fiscal variables responding to debt fits data best
    \item Magnitude of fiscal shocks depend on composition
    \item Fiscal multipliers can have unexpected signs, depending on composition
    \end{itemize}
  \end{block}
  } % end uncover

  \uncover<+-> {
  \begin{block}{Debt target and interactions matter}
    \begin{itemize}
    \item Fiscal responses will depend on \textit{debt target}
    \end{itemize}
  \end{block}
  } % end uncover

\end{frame}

\begin{frame}
  \ft{Related fiscal policy literature}
  \begin{itemize}
  \item<+-> Favero and Montecelli (2005): Deficit feedback rule with \textit{Markov~switching}
    \uncover<.->{
      \begin{itemize}
      \item Switching explains data better
      \item Deficits switch between \textit{active} and \textit{passive} regimes
      \end{itemize}
    }
  \item<+-> Chung, Davig, Leeper (2007): Switching in monetary \& fiscal policy
    \uncover<.->{
      \begin{itemize}
      \item \textit{Switching} in fiscal policy can adversely affect stabilizing impact of monetary policy
      \end{itemize}
    }

  \item<+-> Ko and Morita (2013): Switching in government expenditures and taxes in Japan    
  \item<+-> Bohn (1998, 2005): Deficit responds (as it should) to debt/GDP.
  \item<+-> Jones (JME, 2001): Fiscal stabilizers do little to reduce volatility, recession duration

  \end{itemize}
\end{frame}

\begin{frame}
  \ft{Fiscal behavioral relationships}
  \begin{block}{Evolution of fiscal variables}
    \bdm f_t = \rho_f(s_t) f_{t-1} + \left[1-\rho_f(s_t)\right] f^*_t, \edm
    \bdm f^*_t = \bar{f}(s_t) + \psi_f(s_t) x_t + \gamma_f(s_t) \left[ b_{t-1} - \bar{b}(s_t) \right] + u_{f,t},\edm
    \begin{center}$s_t \in \{1,2,..M\}$ is fiscal \textit{regime}... more later...\end{center}
  \end{block}

  \begin{block}{Fiscal variables}
    \bdm f_t \in \{\tau_t, n_t, g_t \} \edm
    (1) \textit{Tax rate},$~$ (2) \textit{Net transfers / GDP},$~$ (3) \textit{Gov exp / GDP}
  \end{block}
  
  \begin{block}{Notation}
    \begin{scriptsize}
    \begin{tabular}{llcll}
      $f_t$ & Fiscal variable &  & $x_t$ & Output gap \\ 
      $f_t^*$ & Time $t$ target for $f_t$ & & $\rho_f(s_t)$ & Persistence of $f_t$ \\
      $\bar{f}(s_t)$ & \textit{Long-run} target for $f_t$ & & $\psi_f(s_t)$ & Feedback on output gap \\
      $b_t$ & Debt / GDP ratio & & $\gamma_f(s_t)$ & Feedback on debt/GDP \\
      $\bar{b}(s_t)$ & \textit{Long-run} target for debt/GDP & & $u_{f,t}$ & Innovations to $f_t$ \\
    \end{tabular}
    \end{scriptsize}
  \end{block}
  
\end{frame}

\begin{frame}
  \ft{Stochastic terms}
  \begin{block}{Evolution of stochastic shocks}
    \bdm u_{f,t} = \phi_{f,\tau}(s_t) e_{\tau,t} + \phi_{f,n}(s_t) e_{n,t} + \phi_{f,g}(s_t) e_{g,t} \edm
    \bdm e_{f,t} = \alpha_f(s_t) e_{f,t-1} + \sigma_f(s_t) v_{f,t}, ~~ v_{f,t} \sim N(0,1) \edm
  \end{block}

  \begin{block}{Notation}
    \begin{itemize}
    \item $\phi_{f,f'}(s_t)$: captures co-dependence of fiscal policy shocks
    \item $\phi_{f,f}(s_t) \equiv 1$
    \item $v_{f,t}$: iid shock to fiscal variable $f_t$
    \item $\sigma_f(s_t)$: standard deviation of iid shock to fiscal variable $f_t$
    \end{itemize}
  \end{block}
    
\end{frame}

\begin{frame}
  \ft{Deficits and Debt}
  \uncover<+->{\begin{block}{Evolution of primary deficit}
    \bdm d_t = \rho_d(s_t) d_{t-1} + \left[1-\rho(s_t)\right] d^*_t \edm
    \bdm d^*_t = \bar{d}(s_t) + \psi_d(s_t) x_t + \gamma_f \left[b_{t-1} - \bar{b}(s_t)\right] + u_{d,t} \edm
    \bdm u_{d,t} = \phi_{d,\tau}(s_t) e_{\tau,t} + \phi_{d,n}(s_t) e_{d,t} + \phi_{d,g}(s_t) e_{g,t} + e_{d,t} \edm
    \bdm e_{d,t} = \alpha_d(s_t) e_{d,t-1} + \sigma_d(s_t) v_{d,t}, ~~ v_{d,t} \sim N(0,1) \edm
  \end{block}}

  \uncover<+->{\begin{block}{Evolution of debt}
    \vspace*{-0.5pc}\begin{displaymath} 
      \begin{array}{ll} \mbox{Nominal terms:}~~ & B_t = (1+r_t) B_{t-1} + D_t - (M_t - M_{t-1}) \\
        \mbox{As \% of GDP:} & b_t = \frac{1+r_t}{1+y_t} b_{t-1} + d_t - m_t 
      \end{array}
    \end{displaymath}
    \vspace*{-1pc}\begin{itemize}
      \item $y_t$: \textit{Quarterly} nominal GDP growth 
      \item $r_t$: Government borrowing rate
      \item $m_t \equiv (M_t - M_{t-1})/Y_t$: Seigniorage / GDP 
    \end{itemize}
  \end{block}}
\end{frame}

\begin{frame}
  \ft{Fiscal Targets}
  \uncover<+->{\begin{block}{Four long-run fiscal targets to estimate:}
  \bi
  \item $\bar{\tau}(s_t)$: Income tax rate
  \item $\bar{n}(s_t)$: Net transfers / GDP 
  \item $\bar{g}(s_t)$: Government expenditures / GDP
  \item $\bar{b}(s_t)$: Debt / GDP 
  \ei
  \end{block}}

  \uncover<+->{\begin{block}{Implied long-run deficit target}
  For a given regime, set $b_t = b_{t-1} = \bar{b}(s_t)$, then
    \bdm \bar{d}(s_t) = \frac{\bar{y} - \bar{r}}{1+\bar{y}} \bar{b}(s_t) + \bar{m}, \edm
    Calibrate $\bar{y}=0.0158$, avg \textit{quarterly} growth rate in nominal GDP;\newline
    $\bar{m}=0.0090$, avg seigniorage (\textit{quarterly $\Delta$}) / GDP ratio;\newline
    $\bar{r}=0.01857$, avg of \textit{quarterly} interest payments / debt.\newline
  \end{block}}
\end{frame}

\begin{frame}
  \ft{Regime switching}
  \uncover<+->{\begin{block}{Description}
  \begin{itemize}
  \item Two fiscal policy regimes, $s_t \in \{1,2\}$
  \item All parameters may take on two values, one for each regime
  \item Each fiscal policy variable can change its,
    \begin{itemize}
    \item long-run magnitude 
    \item use for stabilization (response to $x_t$)
    \item use for balancing long-run government budget (response to $b_t$)
    \item volatility
    \end{itemize}
  \end{itemize}
  \end{block}}

  \uncover<+->{\begin{block}{Exogenous Markov switching}
    \begin{displaymath} P(s_t=j | s_{t-1}=i) = p_{i,j},~~ p_{i,j} \in (0,1)~~ \sum_{j=1}^{M} p_{i,j} = 1 \end{displaymath}
  \end{block}}
\end{frame}

\begin{frame}
  \ft{Data}
  \begin{block}{Federal personal income tax rate ($\tau_t$)}
    \begin{displaymath} \tau_t = \frac{IT_t}{W_t + PRI_t + RI_t + CP_t + II_t} \end{displaymath}
    \vspace*{-1pc}\begin{itemize}
    \item $IT$ is federal personal income tax (NIPA 3.2 Line 3),
    \item $W$ is wages \& salaries (NIPA 1.12 Line 3),
    \item $PRI$ is proprietor's income (NIPA 1.12 Line 9),
    \item $RI_t$ is rental income (NIPA 1.12 Line 12),
    \item $CP_t$ is corporate income (NIPA 1.12, Line 13),
    \item $II_t$ is interest income (NIPA 1.12 Line 12). 
    \end{itemize}
  \end{block}
\end{frame}

\begin{frame}
  \ft{Data}
  \uncover<+->{\begin{block}{Federal net current transfers / GDP ($n_t$)}
    \begin{displaymath} n_t = \frac{TRP_t - TRR_t}{GDP_t} \end{displaymath}
    \vspace*{-1pc}\begin{itemize}
    \item $TRP_t$ is federal current transfer payments (NIPA 3.2 Line 25)
    \item $TRR_t$ is federal current transfer receipts (NIPA 3.2 Line 18)
    \item $GDP_t$ is nominal GDP (NIPA 1.1.5 Line 1)
    \end{itemize}
  \end{block}}

  \uncover<+->{\begin{block}{Government expenditures / Nominal GDP ($g_t$)}
    \begin{displaymath} g_t = \frac{GC_t + GI_t}{GDP_t} \end{displaymath}
    \vspace*{-1pc}\begin{itemize}
    \item $GC_t$ is federal government consumption expenditures (NIPA 3.2 Line 24)
    \item $GI_t$ is federal government gross investment expenditures (NIPA 3.2 Line 44)
    \end{itemize}
  \end{block}}
\end{frame}

\begin{frame}
  \ft{Data}
  \uncover<+->{\begin{block}{Primary deficit / GDP ($d_t$)}
    \begin{displaymath} d_t = (-SG_t - IP_t) / GDP_t \end{displaymath}
    \begin{itemize}
    \item $SG_t$ is net federal government saving (NIPA 3.2 Line 36)
    \item $IP_t$ is federal interest payments (NIPA 3.2 Line 32)
    \end{itemize}
  \end{block}}

  \uncover<+->{\begin{block}{Exogenous budget constraint variables}
    \begin{itemize}
    \item Interest payments / GDP, $~r_t = IP_t / Debt_t$,\newline $~~$where $Debt_t$ is total federal debt. 
    \item Seigniorage / GDP, $~m_t = (M_t - M_{t-1}) / GDP_t$,\newline where $M_t$ is M2 nominal money stock.
    \end{itemize}
  \end{block}}
\end{frame}

\begin{frame}
  \ft{Output gap}
  \begin{block}{Endogeneity problem:}
    \bi
    \item Automatic and (quick acting?) discretionary policy causes output to affect fiscal variables (the effect I am after)
    \item Endogenous feedback: Fiscal policy can have immediate effect on real GDP
    \ei
  \end{block}

  \begin{block}{Instrument for output gap}
    \bi
    \item Run ARDL(4) on own lags, four lags of all variables
    \item Predicted values used as proxy for \textit{exogenously explained} output gap
    \item Similar to Favero and Montecelli (2005)
    \ei
  \end{block}  
\end{frame}

\begin{frame}
  \ft{State / Space Representation}
  \uncover<+->{\begin{block}{State equation}
    \begin{displaymath} \xi_t = F (s_t) \xi_{t-1} + G (s_t) z_t + M (s_t) v_t \end{displaymath}
    \vspace*{-1pc}\begin{itemize}
       \item Endogenous variables: $\xi_t = \left[ \tau_t~ n_t~ g_t~ d_t~ b_t~ e_{\tau,t}~ e_{n,t}~ e_{g,t}~ e_{d,t} \right]'$
       \item Exogenous variables: $z_t = \left[1~ x_t~ y_t~ m_t \right]$
       \item Shocks: $v_t = \left[v_{\tau,t}~ v_{n,t}~ v_{g,t}~ v_{d,t} \right]$
    \end{itemize}
  \end{block}}

  \uncover<+->{\begin{block}{Observation equation}
    \begin{displaymath} w_t = H x_t \end{displaymath}
    Matrix H picks off observed variables
  \end{block}}

  \uncover<+->{\begin{block}{Kim and Nelson procedure}
    \begin{itemize}
    \item Obtain a set of parameter estimates for each regime
    \item Estimate timing of each regime
    \end{itemize}
  \end{block}}
\end{frame}

\begin{frame}
  \ft{My Questions}
  \begin{itemize}
  \item<+-> Have there been changes in regime?  How does a single regime compare to multiple regimes in terms of model fit?
  \item<+-> What is the timing of regime changes?
  \item<+-> How do regimes compare in terms of long-run debt targets?
  \item<+-> How do regimes compare in fiscal variables' roles for stabilization?  Related to long-run debt targets?
  \item<+-> How do regimes compare in fiscal variables' roles for balancing long-run budget?
  \end{itemize}
\end{frame}
  
\begin{frame}
  \ft{Future papers}
  \begin{itemize}
  \item<+-> Could agents with adaptive expectations learn about regime changes?
  \item<+-> In the context of a DSGE, describe time-dynamics of IRF of a fiscal shock, given different regimes.
  \item<+-> In the context of a DSGE, what is the impact of a fiscal shock when knowledge of regime is known, unknown, incorrect?
  \end{itemize}
\end{frame}

\end{document}

